%
%	proposed_design.tex
%
%	Project Details: Proposed Design
%
%	John Hughes and Michael Jean
%	University of Manitoba
%

\section{Proposed Design}

\subsection{Engine Controller}

\subsubsection{Mechanical}
\subsubsection{Hardware}
\subsubsection{Software}

\subsection{Brake Controller}

\subsubsection{Mechanical}
\subsubsection{Hardware}
\subsubsection{Software}

\subsection{Wireless Telemetry}

The ability to communicate with the car wirelessly is a very important
feature of most current FSAE designs. There are various systems in
the University of Manitoba's Formula SAE car that are important to
interface with remotely, including the ECU and the \emph{Data Aquisition
Device} (DAQ\nomenclature{DAQ}{Data Aquisition device, provides and logs
high-performance GPS, accelerometer, and ADC input data}).
Both the ECU and the DAQ hardware provide a wired RS232 interface
to a Windows PC running proprietary software that communicates with
each device. Each software expects to talk directly to the device
through a Windows COM port.

Our goal with the wireless telemetry for the 2009/2010 car is to multiplex
the two serial streams, and send the data wirelessly using an XBee
modem to multiple different receiver laptops using a receiver modem.
A small software driver will split the two data streams apart, and
dump the data into two different virtual serial ports on the PC. The
goal with this is that the driver will allow proprietary software
to run transparently without any modifications. A block diagram of
the telemetry system can be seen in Figure
\ref{fig:Wireless-Telemetry-Overview}.

\subsubsection{Mechanical}
\subsubsection{Hardware}
The telemetry module will be implemented on a custom PCB with the same
microcontroller common to the other modules, the AT90CAN128. In addition to the
common life-support hardware, the telemetry module will also include a dual
RS232 trasceiver chip and 2 DSub9 connectors. The ECU and the DAQ will connect
with the telemetry board with standard 9-pin serial cables. While the RS232
transceiver will interface with the AT90CAN's 2 built-in USART ports, a third
UART will be needed to interface with an XBee Pro wireless modem. For this, the
telemetry module will include an SPI interfaceable UART chip.

The XBee Pro Modems from Digi International require a 3.3V power supply, and
consume at most 215mA of current during transmit. Since the common module
hardware provides power for only 5V devices,
the telemetry module will have a second LDO regulator providing 3.3V.

  \begin{figure}[H]
    \begin{centering}
      \def\antenna{
  -- +(0mm,4.0mm) -- +(2.625mm,7.5mm) -- +(-2.625mm,7.5mm) -- +(0mm,4.0mm)
}

\begin{tikzpicture}[auto, node distance=3cm, draw=black!70, >=stealth']
  \node [block, name=telemetry] {Telemetry};
  \node [block, name=ecu, left of=telemetry, left=0.5cm] {ECU};
  \node [block, name=daq, right of=telemetry, right=0.5cm] {DAQ};
  \node [bus, name=ant, above of=telemetry, below=1.5cm] {};

  \node [bus, name=can1, below of=telemetry, above=1.5cm, label=below:CAN Bus] {CAN Bus};
  \node [bus, name=can2, left of=can1] {};
  \node [bus, name=can3, right of=can1] {};

  \draw [-, line width=3pt] (telemetry) -- (can1);
  \draw [-, line width=3pt] (can1) -- (can2);
  \draw [-, line width=3pt] (can1) -- (can3);

  \draw [<->, thick] (telemetry) -- node[] {RS232} (ecu);
  \draw [<->, thick] (telemetry) -- node[] {RS232} (daq);
  \draw [-, thick] (telemetry) -- node[text width=1.5cm] {XBee Modem} (ant) \antenna;
\end{tikzpicture}
    \end{centering}

    \caption{Wireless Telemetry Overview\label{fig:Wireless-Telemetry-Overview}}
  \end{figure}

  \begin{table}[H]
    \caption{Wireless Telemetry Module
Components\label{tab:Wireless-Telemetry-Module}}
    \begin{centering}
      \begin{tabular}{|c|c|c|}
	\hline 
	Part & Description & Part Number\tabularnewline
	\hline
	\hline
	Wireless Modem & XBee-PRO OEM Module & XBee-PRO\tabularnewline
	\hline 
	RS232 Transciever & Dual RS232 Trasciever & MAX232\tabularnewline
	\hline 
	SPI-UART & Additional SPI interfacable UART chip &
MAX3100\tabularnewline
	\hline 
	3v3 LDO & 300mA Low Dropout Regulator & LT1521\tabularnewline
	\hline
      \end{tabular}
    \end{centering}
  \end{table}

\subsubsection{Software}

\subsection{Driver Interface}

\subsubsection{Mechanical}
\subsubsection{Hardware}
\subsubsection{Software}
