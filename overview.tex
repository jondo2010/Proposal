%
%	overview.tex
%
%	Project Details: Overview
%
%	John Hughes and Michael Jean
%	University of Manitoba
%

\documentclass[english]{report}
%
%	preamble.tex
%
%	Proposal: Document Preamble
%
%	John Hughes and Michael Jean
%	University of Manitoba
%

\setcounter{secnumdepth}{3}
\setcounter{tocdepth}{3}

\usepackage[T1]{fontenc}
\usepackage[latin9]{inputenc}

\usepackage{array}
\usepackage{babel}
\usepackage{fancyhdr}
\usepackage{float}
\usepackage[letterpaper]{geometry}
\usepackage{graphicx}
\usepackage{lastpage}
\usepackage{nomencl}
\usepackage{subfig}
\usepackage{tikz}

% Set Margins
\geometry{verbose,tmargin=1in,bmargin=1in,lmargin=1.5in,rmargin=1in}

\pagestyle{fancy}

\fancyhead[LE,LO]{Draft Proposal}
\fancyfoot[LE,LO]{
  \footnotesize{
    John Hughes, john\_hughes@umanitoba.ca\\
    Michael Jean, michael.jean@shaw.ca
  }
}

\fancyfoot[CE,CO]{\thepage\ of \pageref{LastPage}}
\fancyfoot[RE,RO]{
  \footnotesize{
    \jobname\\
    \today
  }
}

\usetikzlibrary{shapes,arrows,shadows,calc}

\tikzstyle{block} = [ draw=red!50!black!50, % 50% red, 50% black
		      top color=white,
		      bottom color=red!50!black!20,
		      rectangle,
		      very thick,
		      minimum height=3em,
		      minimum width=6em,
		      inner sep=0.5em,
		      drop shadow,
		      text width=2cm,
		      text centered]
\tikzstyle{bus} = [coordinate]
\tikzstyle{input} = [coordinate]
\tikzstyle{output} = [coordinate]

\setlength{\parskip}{12 pt}

\usepackage[
  unicode=true,
  pdfusetitle,
  bookmarks=true,
  bookmarksnumbered=true,
  bookmarksopen=false,
  breaklinks=false,
  pdfborder={0 0 0},
  backref=false,
  colorlinks=false
] {hyperref}

\begin{document}

\section{System Overview}

The Formula SAE vehicle is a performance car built with the primary goal of doing well in the 
dynamic events at the yearly competitions. These events test the vehicles abilities in acceleration, 
braking, and handling.

The vehicle as a whole is primarily a mechanical device, but carries several critical electronic 
control systems. Three key mechanical systems that can benefit from electronic control are

\begin{itemize}
\item the \emph{internal combustion engine} (ICE\nomenclature{ICE}{Internal Combustion Engine}); 
\item the \emph{braking system}; and
\item the \emph{suspension system}.
\end{itemize}

\section{Internal Combustion Engine}

The ICE is a \emph{Honda CBR600F4i}, a 600 cm$^3$ super-sport class engine with an internal 6-speed 
chain driven transmission. The engine has a set of sensors attached to it, including

\begin{itemize}
\item an O$_{2}$ sensor;
\item a Manifold Absolute Pressure (MAP) sensor; \nomenclature{MAP}{Manifold Absolute Pressure}
\item an oil pressure sensor;
\item a water temperature sensor;
\item a throttle position sensor;
\item a gear position sensor;
\item cam and crank position sensors; and
\item wheel speed sensors.
\end{itemize}

Several components of the internal combusion engine require electronic control, or could benefit from 
the introduction of electronic control, namely

\begin{itemize}
\item timing signals for the fuel injectors must be generated;
\item firing signals for the spark coils must be generated;  
\item a starter signal for the starter solenoid must be provided;
\item the clutch and gear selection levers must be intelligently actuated to change gears; and
\item the intake valve must be opened or closed as driving conditions warrant.
\end{itemize}

\nomenclature{ECU}{Engine Control Unit}
A specialized third-party component called the \emph{Engine Control Unit} (ECU) reads O$_{2}$
and MAP sensor outputs and adjusts fuel injector and spark coil timings to keep the engine running 
smoothly. The ECU also features a traction control system that monitors wheel slip and cuts spark 
signal to help give traction when one of the drive wheels is slipping. The particular model of
ECU used by the Formula SAE car is the S80Pro from DTAFast\cite{s60pro}.

The ICE is started by energizing a \emph{starter solenoid}. This solenoid relays a large electric current 
to the \emph{starter motor}, which turns over the engine and causes it to start. 

The ICE has an internal 6-speed manual transmission. The first, fifth, and sixth gears have been 
removed to reduce weight. The vehicle is not operated at speeds that would benefit from the presence 
of the fifth or sixth gear, and launching from second gear provides the best acceleration. A pre-tensioned 
spring lever controls the clutch. The spring keeps the clutch engaged when there is no force on the lever.
As force is applied to the lever, the clutch is gradually disengaged. The relationship between
lever position the distance between the clutch plate and flywheel is non-linear. Another
pre-tensioned spring lever controls the gear selection. The gear selector may be rotated in either 
direction from its rest point. Up-shifting is accomplished by rotating the lever in one direction, 
while down-shifting is accomplished by rotating the lever in the opposite direction. 

\section{Braking System}

\section{Suspension System}

Variables like fuel injection timing, spark coil firing rate, and  are determined with closed-loop
feedback systems 

The ICE, a Honda CBR600 F4i, is controlled by an off-the-shelf S80Pro \emph{Engine Control Unit} 
(ECU\nomenclature{ECU}{Engine Control Unit}) from DTAFast\cite{s60pro}. The ECU provides
timing signals to the fuel injectors and spark coils, enabling smooth operation of the ICE. 
A feedback path to the ECU from a wide-band oxygen sensor downstream of the combustion chamber 
allows the controller to sense the amount of residual oxygen in the exhaust, and allows closed-loop 
control of the combustion process. The ECU also reads several tempurature, pressure, and rotational 
velocity sensors from the engine.

\end{document}
