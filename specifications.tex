%
%	specifications.tex
%
%	Project Details: Specifications
%
%	John Hughes and Michael Jean
%	University of Manitoba
%

\chapter{Specifications}

Specifications on what will be required from the end result deliverables of this project are described in the following sections.

\section{General Requirements}

\subsection{Hardware}
All designs will need to be implemented in hardware and tested and debugged on the 2010 Formula SAE vehicle.

\paragraph{Elemental Resistance}
All electronic components must be designed with elemental enclosure in mind. Sealed header connectors will need to be used, and the PCB boards secured in a water resistant enclosure. The components must also be able to withstand the vibrations, G forces, and heat on the Formula SAE vehicle.

\paragraph{Programming/Debugging ease}
All microcontroller systems must be easy to program and debug. Programming headers must be available on the PC board allowing for in-place programming using a laptop and USB programmer.

\subsection{Software}

\paragraph{Code Reuse}
Code for all modules should be reused as much as possible to reduce the complexity of the software design.

\paragraph{Runtime Modifiable Operation}
All parameters used by the software that might need to be modified during testing, tuning, and debugging must be modifiable at runtime either through the driver interface, or through the CAN debugger.

\section{Engine Module}
The engine module must be able to
\begin{enumerate}
 \item start the engine by energizing the starter solenoid;
 \item activate and deactivate launch control on the ECU;
 \item send appropriate shift cut signals to the ECU;
 \item precisely control the position of the clutch;
 \item actuate the clutch from fully engaged, to fully disengaged in less than 150 ms.
 \item feather the clutch during low throttle launch; and
 \item actuate the variable intake manifold (timing requirements to be determined).
\end{enumerate}

\section{Brake Module}
The brake module must be able to
\begin{enumerate}
 \item absolutely position the brake bias bar;
 \item adapt to changes in assembly position, etc, by recalibration; and
 \item detect pressure on the brake pedal.
\end{enumerate}

\section{Wireless Telemetry Module}
The wireless telemetry module must be able to
\begin{enumerate}
 \item transparently transport data bidirectionally with the ECU over the ECUs serial
connection, and a laptop running the DTASwin tuning software;
 \item broadcast telemetry data to one or more laptops running the Race Technology
software. The serial protocol will be that of Race Technology, and data will
come primarily from the DAQ, but also from other sensors; and
 \item send messages over the CAN network with data from the DAQ, such as accelerometer data.
\end{enumerate}

\section{Driver Interface Module}
The driver interface module must be able to
\begin{enumerate}
 \item change vehicle dynamics mode;
 \item alter vehicle dynamics mode parameters on the fly;
 \item permanently store vehicle dynamics parameters;
 \item display status of
  \begin{itemize}
    \item wireless telemetry (Signal strength, association status);
    \item engine parameters such as RPM, water temp, oil pressure, and warning messages when these are out of range;
    \item vehicle speed (Either wheel speed, or from the GPS); and
    \item current gear.
  \end{itemize}
 \item start the engine;
 \item alter module-specific parameters, such as shift timing constants; and
 \item display team logo, and certain team sponsor logos on startup.
\end{enumerate}